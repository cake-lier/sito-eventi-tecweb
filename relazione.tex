\documentclass[a4paper, 12pt]{report}

\usepackage{amsmath}
\usepackage{textcomp}
\usepackage[italian]{babel}
\usepackage[a4paper, total={17cm, 24cm}]{geometry}
\usepackage[format=plain,labelfont=bf,up,textfont=normal,up,justification=justified,singlelinecheck=false,skip=0.01\linewidth]{caption}

\title{Relazione progetto di Tecnologie Web}
\author{Matteo Castellucci, Giorgia Rondinini}
\date{\today}

\begin{document}
	
\section*{Relazione progetto di Tecnologie Web}

\textbf{Data}: 09/01/2020\newline
\textbf{Autori}: Matteo Castellucci, matricola 0000825436; Giorgia Rondinini, matricola 0000825091\newline
\newline
La progettazione del sito web è iniziata con una sessione di design partecipativo con un gruppo di possibili utenti, chiedendo loro quali funzionalità avrebbero voluto vedere nel sito, ma soprattutto quali avrebbero voluto più rapidamente accessibili. In seguito a ciò si è creata una serie di \textit{mock-up}, poi sottoposti alla valutazione di un ulteriore \textit{focus group}.\newline Il confronto con i \textit{focus group} è stato mirato soprattutto alle sezioni del sito rivolte agli utenti consumatori.\newline\newline
Dalla prima sessione di design partecipativo è emerso che le funzionalità di maggiore interesse per gli utenti erano: la ricerca e suddivisione degli eventi disponibili per categoria, data, luogo, organizzatore, la visualizzazione di eventi popolari o consigliati, un meccanismo per il carrello acquisti elastico che permettesse di modificare facilmente il numero di biglietti da acquistare e che fosse facile da raggiungere soprattutto dopo l'inserimento di biglietti, un’ampia scelta di metodi di pagamento e un’interfaccia che permettesse di individuare facilmente le informazioni su un dato evento, in maniera quanto più possibile trasparente ed intuitiva.\newline
In seguito alla sessione di design partecipativo sono stati prodotti dei \textit{mock-up} animati per le principali schermate del sito, sia in versione mobile che in versione desktop, usando il software ``Adobe XD".\newline
I \textit{mock-up} sono stati prodotti considerando quanto emerso dall'incontro con il primo \textit{focus group}, ma anche immedesimandosi in un possibile utente per le decisioni riguardanti la struttura del sito e quelle riguardanti il \textit{layout} delle pagine. Si è prestata particolare attenzione durante questa fase della progettazione ai collegamenti tra una schermata e l’altra del sito, cercando di renderlo il più ergonomico e facilmente navigabile possibile.\newline
Il secondo \textit{focus group} ha approvato la maggior parte dei \textit{mock-up}, criticando o commentando soltanto alcuni particolari: il passaggio dalla schermata di login a quella di registrazione è stato giudicato poco chiaro, il comportamento del link in alto a sinistra di ritorno alla pagina principale non facilmente intuibile, il collasso del menù principale nella versione \textit{mobile} non particolarmente ergonomico, il menù dell'area personale sempre nella versione \textit{mobile} brutto e la struttura del carrello ridondante. È stata inoltre ribadita l’importanza della ricerca per organizzatore e per categoria di evento e la ricerca per titolo parziale.\newline\newline
Dato che i \textit{mock-up} erano stati perlopiù approvati si è scelto di non correggerli secondo quanto emerso dalle richieste del secondo \textit{focus group} ma di operare direttamente sul sito web.
Alcune funzionalità (ad esempio la visualizzazione degli iscritti a un evento), soprattutto per gli utenti organizzatori, non sono state implementate, non essendo certi che fossero legali secondo il GDPR e il Codice Privacy. Si è infatti cercato, per quanto possibile, di implementare le basilari richieste legali,
come ad esempio la presenza dei termini d'uso e lo status dell'acquisto dell'utente. Inoltre, il sistema è stato progettato e sviluppato in maniera tale da essere più aperto possibile all'introduzione di nuove funzionalità con difficoltà minima.

\end{document}